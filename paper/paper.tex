%-----------------------------------------------------------------------------
%
%               Template for sigplanconf LaTeX Class
%
% Name:         sigplanconf-template.tex
%
% Purpose:      A template for sigplanconf.cls, which is a LaTeX 2e class
%               file for SIGPLAN conference proceedings.
%
% Guide:        Refer to "Author's Guide to the ACM SIGPLAN Class,"
%               sigplanconf-guide.pdf
%
% Author:       Paul C. Anagnostopoulos
%               Windfall Software
%               978 371-2316
%               paul@windfall.com
%
% Created:      15 February 2005
%
%-----------------------------------------------------------------------------


\documentclass[preprint,11pt]{sigplanconf}

% The following \documentclass options may be useful:

% preprint      Remove this option only once the paper is in final form.
% 10pt          To set in 10-point type instead of 9-point.
% 11pt          To set in 11-point type instead of 9-point.
% authoryear    To obtain author/year citation style instead of numeric.

\usepackage{amsmath}
\usepackage{listings}
\usepackage{color}

\definecolor{mygreen}{rgb}{0,0.6,0}
\definecolor{mygray}{rgb}{0.5,0.5,0.5}
\definecolor{mymauve}{rgb}{0.58,0,0.82}
\definecolor{mygray2}{rgb}{0.95,0.95,0.95}

\lstset{ %
  backgroundcolor=\color{mygray2},   % choose the background color; you must add \usepackage{color} or \usepackage{xcolor}
  basicstyle=\footnotesize,        % the size of the fonts that are used for the code
  breakatwhitespace=false,         % sets if automatic breaks should only happen at whitespace
  breaklines=true,                 % sets automatic line breaking
  captionpos=b,                    % sets the caption-position to bottom
  commentstyle=\color{mygreen},    % comment style
  deletekeywords={...},            % if you want to delete keywords from the given language
  escapeinside={\%*}{*)},          % if you want to add LaTeX within your code
  extendedchars=true,              % lets you use non-ASCII characters; for 8-bits encodings only, does not work with UTF-8
  %% frame=single,                    % adds a frame around the code
  keepspaces=true,                 % keeps spaces in text, useful for keeping indentation of code (possibly needs columns=flexible)
  keywordstyle=\color{blue},       % keyword style
  language=Octave,                 % the language of the code
  morekeywords={*,...},            % if you want to add more keywords to the set
  numbers=none,                    % where to put the line-numbers; possible values are (none, left, right)
  numbersep=5pt,                   % how far the line-numbers are from the code
  numberstyle=\tiny\color{mygray}, % the style that is used for the line-numbers
  rulecolor=\color{black},         % if not set, the frame-color may be changed on line-breaks within not-black text (e.g. comments (green here))
  showspaces=false,                % show spaces everywhere adding particular underscores; it overrides 'showstringspaces'
  showstringspaces=false,          % underline spaces within strings only
  showtabs=false,                  % show tabs within strings adding particular underscores
  stepnumber=2,                    % the step between two line-numbers. If it's 1, each line will be numbered
  stringstyle=\color{mymauve},     % string literal style
  tabsize=2,                       % sets default tabsize to 2 spaces
  title=\lstname                   % show the filename of files included with \lstinputlisting; also try caption instead of title
}
%%
%% Coq definition (c) 2001 Guillaume Dufay
%% <Guillaume.Dufay@sophia.inria.fr>
%% with some modifications by J. Charles (2005)
%%
\lstdefinelanguage{Coq}%
                  {morekeywords={Variable,Inductive,CoInductive,Fixpoint,CoFixpoint,%
                      Definition,Example,Lemma,Theorem,Axiom,Local,Save,Grammar,Syntax,Intro,%
                      Trivial,Qed,Intros,Symmetry,Simpl,Rewrite,Apply,Elim,Assumption,%
                      Left,Cut,Case,Auto,Unfold,Exact,Right,Hypothesis,Pattern,Destruct,%
                      Constructor,Defined,Fix,Record,Print,Proof,Induction,Hint,Hints,Exists,
                      let,in,Parameter,Split,Red,Reflexivity,Transitivity,if,then,else,Opaque,%
                      Transparent,Inversion,Absurd,Generalize,Mutual,Cases,of,end,Analyze,%
                      AutoRewrite,Functional,Scheme,params,Refine,using,Discriminate,Try,%
                      Require,Load,Import,Scope,Set,Open,Section,End,match,with,Ltac,%
                      exists,forall,fix,fun
                    },%
                    sensitive, %
                    morecomment=[n]{(*}{*)},%
                    morestring=[d]'',%
                    literate={=>}{{$\Rightarrow$}}1 {>->}{{$\rightarrowtail$}}2{->}{{$\to$}}1
                    {\/\\}{{$\wedge$}}1
                    {|-}{{$\vdash$}}1
                    {\\\/}{{$\vee$}}1
                    {~}{{$\sim$}}1
                    %{<>}{{$\neq$}}1 indeed... no.
                  }[keywords,comments,strings]%


\begin{document}

\special{papersize=8.5in,11in}
\setlength{\pdfpageheight}{\paperheight}
\setlength{\pdfpagewidth}{\paperwidth}

\conferenceinfo{6.888 '15}{March 19, 2015, Cambridge, MA, U.S.A.} 
\copyrightyear{2015} 
\copyrightdata{978-1-nnnn-nnnn-n/15/03} 
\doi{nnnnnnn.nnnnnnn}

% Uncomment one of the following two, if you are not going for the 
% traditional copyright transfer agreement.

%\exclusivelicense                % ACM gets exclusive license to publish, 
                                  % you retain copyright

%\permissiontopublish             % ACM gets nonexclusive license to publish
                                  % (paid open-access papers, 
                                  % short abstracts)

\titlebanner{6.888: Certified Systems Software 2015}        % These are ignored unless

\preprintfooter{Framework to prove 8-bit embedded
  software systems on the Intel 8051 series} % 'preprint' option specified.

\title{V-EmRALD}
\subtitle{Verified Embedded Realtime Assembly Language Description for
 8-bit Intel 8051 Semantics}

\authorinfo{Jeremy Rubin}
 {jlrubin@mit.edu}

\maketitle

\begin{abstract}
  Microcontrollers are used widely in a wide range of devices and
  machines.  There are many life critical applications of
  microcontrollers including medical equipment, transport, and
  communications. It is therefore of great importance to ensure the
  proper functionality of such devices. Many microcontrollers are
  simple, preferring to rely on a tried and true 8-bit architecture
  without a full operating system so that they may more easily reason
  about real-time responses to events. For instance, it could be
  disastrous to have a garbage collection pause while trying to apply
  the brakes of a car. However, this simplicity comes at a cost;
  without higher level constructs a programmer must manually write and
  check a lot more code. Furthermore, it is hard for a programmer to
  verify that the assembly they wrote is correctly translated into the
  binary loaded onto chip.


  V-EmRALD is a new framework for programming 8-bit Intel 8051 series
  microcontroller which includes a verified assembler, 8051 simulator, and
  higher level constructs to help programmers formalize the behavior
  and run time of control loops. We are currently building the
  prototype of V-EmRALD.
\end{abstract}

%% \category{CR-number}{subcategory}{third-level}

% general terms are not compulsory anymore, 
% you may leave them out
%% \terms
%% term1, term2

%% \keywords
%% keyword1, keyword2

\section{Introduction}
Modern machines often rely on a combination of mechanical systems,
electrical systems, sensors, and software. The advent of embedded systems
has allowed for much more accurate and complicated designs to be built because
they allow for tighter feedback loops as well as convenient ways to interface
between disparate systems.

Often times these embedded systems are built with simple
microcontrollers ($\mu Cs$) such as 8-bit because a more powerful
computers are not needed, energy-inefficient, and are harder to reason
about for real-time applications. In some sectors, Real-Time Operating
Systems (RTOS) have become popular due to programming ease, but there
are many applications where a lighter weight solution is needed. Many
systems which require light weight microcontrollers also happen to be
life-critical. Unfortunately, there have been many instances where the
life-critical nature of this role was made all too clear with
catastrophic failures.

V-EmRALD is an effort to reduce the likelihood of programmer errors as
well as toolchain errors when working with 8-bit Intel 8051 series
microcontrollers. This chip was chosen because the author is familiar
with them and they are very widely deployed. V-EmRALD, once
implemented, will be composed of a trusted simulator, a verified
assembler, and the utilities which assist programmers in formally
reason about behavior of control loop programs. Behavior can be
characterized by several properties: return values, execution time,
pin output, and non interference. Return values are defined as a
separation logic over assembly chunks. The basic notion would be that
after execution between two labels, some properties are
preserved. V-EmRALD will initially target a simplified separation
logic which for many embedded applications is enough. This logic will
be solely be based on register restoring after calls and writing to a
limited space. This restriction can be lifted in the future. Execution
time will be defined as the number of cycles required to run from a
label to another label. Pin output will be defined as the state of the
pins on the chip, and can be modeled as a separation logic as well
(pins are like memory). There will be special utilities to deal with
the input and output nature of pins. Non interference statements are
meta statements on behavior which will capture behavior properties
based on different input patterns. Examples could include ensuring
that a single control handler is executed when it is in critical
modes, a handler is not starved for more than N cycles, or a token for
control over some IO pins is explicitly passed from handler to
handler.

This system will have many limitations which may affect its real-world usefulness,
which will be expounded in the discussion.

\section{Motivation}
There are several concrete examples of embedded system failures which
motivate V-EmRALD.
\subsection{Medical Devices}
The Therac-25 was a radiation therapy machine which lethally overdosed
6 people \cite{therac}. While a large portion of the of the blame for
this could be placed on the lack of hardware failsafes, it was
ultimately unverified software which caused the deadly malfunction.

Another interesting medical device is the pacemaker. A 2004 study saw
that approximately 1/1,200 pacemaker re-programmings loaded faulty
programs onto the device, which could cause a patient's heart rate to
elevate to 185 beats per minutes. The fault was due to multiple
timer interrupts being enabled at the same time, when only one was
supposed to be enabled at any given time \cite{pacemaker}.
\subsection{Transport}
Automobiles are also a large concern with regards to embedded systems
as they rely on multiple for the core functionality of the
vehicle. Over the past decade, there have been a number of
bugs and errors caused -- or suspected to be caused -- by faulty
software.  For instance, several major automobile manufacturers --
Nissan, Honda, and Subaru -- recalled vehicles for faulty embedded
systems which could deploy airbags during normal vehicle operation or
start the car at random while parked \cite{cars}.

\subsection{Communications}
Networked mobile devices use small hardware secure chips called
subscriber identification module cards (SIM cards) to reliably
identify clients connecting to the network. They contain a $\mu C$
which manages the processing of cellular data, such as encryption and
signatures for requests from a certain number (such as send a
text-message). Recently, exploits have been found that allow a remote
attacker to take control of the SIM card and issue illegitimate
requests \cite{sim}.
\section{Related Work}
Several projects have related goals:

\subsubsection{Bedrock}
Bedrock is a project which facilitates the formal verification of low
level code. It is implemented as a Coq Library, but it provides
significant enough extensions and capabilities that it is more aptly
described as a language for separation logic. Bedrock provides significant
instruction on the implementation of a separation logic system\cite{bedrock1}\cite{bedrock2}.
\subsubsection{Reflex}
Reflex is a project which makes a verified message passing kernel that
can interpret a set of user defined rules about non interference and
causality. Reflex is not intended for embedded systems, but the model
of non interference and causality is instructive for a style in which
one might prove properties about an embedded system \cite{reflex}. For example, one
such property could be that seems natural to try to prove in a Reflex
style could be that at least 100 cycles pass in between asking for and
reading a value from an Analog Digital Converter.
\subsubsection{Correctness Proofs for Device Drivers in Embedded Systems}
Duan and Regher\cite{drivers} discuss a system which can verify low-level device drivers.
Their approach allows for separately clocked components interactions to be modeled and verified.
They use a pre-existing ARMv4 simulator and develop a formal semantics for interacting with UART. Their proofs are all manual,
and they hope to build proof automation on top of their proofs for UART.
\subsubsection{RockSalt}
This effort derives heavily from RockSalt. RockSalt was a project to
write a formally proven version of Native Client(NaCl), Google's tool
for generating and checking untrusted binaries which can then be
safely run with confidence they will never escape the sandbox
\cite{rocksalt}.

A major part of the RockSalt contribution was an excellent model of
Register Transfer Language on top of which they modeled MIPS and x86
processors. The ability to end-to-end check properties of the binary
is a powerful tool in guaranteeing functional correctness. Their
effort is very instructive in the implementation of V-EmRALD.
\subsubsection{Model Checking Software for Microcontrollers}
This project formally models an 8-bit Atmega microcontroller.
The project reasons about assembly code, but is designed to have proof
automation for high level theorems from C, C++, and other programming langauges.
There does not seem to be facilities for run time verifition \cite{mcsm}.

\section{Overview}

The core of a V-EmRALD checkable program is a single control loop. The
control loop will be a terminating sequence of 
polling pins and and handling conditions. All of the handlers can be
padded out such that the loop takes a constant amount of time. The
terminating sequence will then be looped forever.
\begin{lstlisting}[caption={The control loop here first polls port 1, then checks pin 0 and jumps to the appropriate handler.
      Some more complicated logic is also shown which will handle
      device 3 if both pin 1 or 2 are high, device 2 if pin 1 is high
      and pin 2 is low, and device 1 if pin 2 is high and pin 1 is
      low.},
  label={lst:ControlLoop}]
  loop:
  mov a, P1     
  jb a.0, HandleDevice0
  jnb a.1, skip1
  jb a.2, HandleDevice3
  ljmp HandleDevice2
  skip1:
  jb a.2, HandleDevice1
  ljmp loop
\end{lstlisting}

Before running the loop, a trusted prelude -- a small piece of
assembly -- will turns off 8051 features such as interrupts and
otherwise set up the $\mu C$.

There will be facilities to prove various properties about assembly.

For example, the above control loop would greatly benefit from proofs
that the handler priorities are correct.  There are some neat tricks
which could be used to lighten the proof-burden of pin read non
determinism. One such trick could be, if the hardware designer knows
that the pin being read is a trigger, it can be marked as high across
the entire trace. More nuanced theorems would be required for stateful
loops.
\vfill
\begin{lstlisting}[language=Coq, caption=
    {A couple examples of theorems are potentially easier to read and
      reason about then the raw assembly code of control
      loop in Listing~\ref{lst:ControlLoop}},
  label={lst:ControlLoopTheorems}]

Theorem ctl_lp_p1_0_hi: forall iotrace,
  at_all_times iotrace
  (fun t => read t P1.0 = HIGH) ->
  elem HandleDevice0
  (trace prog iotrace) /\
  !(elem HandleDevice3
  (trace prog iotrace) \/
  elem HandleDevice2
  (trace prog iotrace) \/
  elem HandleDevice1
  (trace prog iotrace)))

Theorem ctl_lp_p1_210_hi_lo_lo: forall iotrace,
  at_all_times iotrace
  (fun t => read t P1.0 = LOW /\
  read t P1.1 = LOW /\
  read t P1.2 = HIGH) ->
  elem HandleDevice2
  (trace prog iotrace) /\
  !(elem HandleDevice0
  (trace prog iotrace) \/
  elem HandleDevice3
  (trace prog iotrace) \/
  elem HandleDevice1
  (trace prog iotrace))
\end{lstlisting}

Theorem ctl\_lp\_p1\_0\_hi in Listing~\ref{lst:ControlLoopTheorems} states that for some user defined program, if P1.0 is high at
all points, then only device 0 will be handled. A more robust, but
perhaps more difficult proof would rely on the pin taking on either
value at all reads, but this may not be accurate given peripheral information.
Theorems about execution time verification could work by having each
instruction executed under a trace and adding the number of cycles it
takes under a trace. The run time can be taken between any two labels
-- the code between the labels must provably terminate though. For instance, 
theorems about proper nop padding and runtime length can be shown.

\begin{lstlisting}[language=Coq, caption=
    {If a segment of code is properly padded then all all executions
      take the same amount of time. The definition of proper\_padded
      is omitted, but it performs some branch expansion},
  label={lst:NopPad}]
Defintion proper_padded c : bool := ...
Theorem loop_const_len: exists L,
        (forall (c:asm) (t:iotrace),
        proper_padded c = true ->
        L = length (trace (run c iotrace))).
\end{lstlisting}
Theorem loop\_const in Listing~\ref{lst:NopPad} states that their
exists a constant amount of time L such that under any io trace (ie,
values read from pins) during execution, proper\_padded c implies that
the program c runs a fixed length number of cycles. This means that a
program checked with proper\_padded can be considered to run a fixed
length amount of time under any trace.


In order to verify such properties, the simulator generates the state
spaces for the 8051 over a given execution and pin trace, generating
an execution trace.  The parser checks over the 8051 binary and
ensures that no unsafe instructions are present, such as setb EA,
which would enable interrupts.  These implementations are heavliy
derived from RockSalt \cite{rocksalt}.
\section{Implementation and Progress}
V-EmRALD is very much a work in progress. Currently, I am working on
getting the 8051 assembly to RTL translation functional. This involves
modifying the RockSalt code to fit the Intel Architecture, as well as
augmenting on some additional features so as to be able to reason
about things like timing. This will likely be a somewhat large
challenge to figure out what the correct semantics are \emph{exactly}
for the added traces. In order to expediate this process, I will
initially focus on a subset of approximately 10 mnemonics which
capture a large swath of 8051 behavior (ie, mov, add, jmp, lcall, and
ret cover a lot of behavior).

Once the simulator is complete, it will be fairly straightforward to
implement the parser based on the Rocksalt derivative method as it is
essentially just punching in a bunch of 1's and 0's from the Intel 8051
manual.

The next challenge will be to implement an assembler. I plan to make
this assembler using Coq's Notation feature. The assembly will likely
be a little bit non standard as it will need to support some special
types of labels, and may not support certain features standard in
assemblers such as .equ directives, because such things can be
accomplished in the coq environment as easily. Making an assembler in
Coq may have some other benefits in aiding later verifiable macro
development \cite{coqasm}.

Once this foundation is laid, then I will be in good shape to start
formally showing the semantics of certain programs. This will again
present a challenge in developing a semantics which are not only
useful to embedded systems engineers, but also are amenable to proof
automation.

Once the above tasks are completed, I will then implement as much of
the ISA as possible.

If time permits, my final contribution would be in developing a
test-jig so that I can verify the 8051 similator against
real hardware.
\section{Discussion}
\subsection{Usability}
V-EmRald cannot prevent all bugs. In order to test the effectiveness
of such a system, it is imperative to run a user study to see if such
a model helps. One such user study could testing and timing users on their
ability to complete the following tasks with and without V-EmRALD (given a small
hardware test-bench):
\begin{enumerate}
\item
  Write a program which makes an LED blink with a certain frequency
\item
  Write a program which writes a copy of an array into a specific
  memory location, maps $f(x) = x+1$ over the array, without modifying
  any other memory.
\item
  Write a program which when a push button is hit, turns one LED on at
  half brightness, and another LED at half brightness otherwise.
\end{enumerate}
These tests would determine if V-EmRALD improves a programmer's
ability to reason about runtime, non-interference, and memory safety.
\subsection{Limitations}
The current plan for V-EmRALD fails to address several key
embedded system properties.
\subsubsection{Interrupts}
Interrupts are explicitly required to be disabled because they would
add a lot of additional complexity to the system. One could potentially,
using an 8051 timers, devise a scheme as follows:
\vfill
\begin{lstlisting}[caption={The timer ISR is used to gaurantee a bound on the amount of time interrupt might be executing.}]
  timer isr:
  disable interrupts
  go to interrupts_over
  set timer to priority high with size 200 cycles
  set all other interrupts priority low
  enable interrupts
  some interruptible loop
  interrupts_over:
\end{lstlisting}
Such a scheme allows for a provably bounded window on interrupts; this is of dubious value given that polling would also accomplish
a similar goal.
\subsubsection{Hardware Interfaces}
Embedded systems fundamentally interface with hardware, which can have
difficult to understand behavior.  V-EmRALD does not do much in the
way of helping a programmer verify that they understood their external
hardware correctly. However, one exciting possibility would be the
development of semantics for other peripheral chips.  This would allow
programmers to verify a deeper model of their designs. However, this
does not help with unreliable hardware such as sensors or power supply.






%% \appendix
%% \section{Appendix Title}

%% This is the text of the appendix, if you need one.

%% \acks

%% Acknowledgments, if needed.

% We recommend abbrvnat bibliography style.
% The bibliography should be embedded for final submission.
\bibliographystyle{abbrv}
\begingroup
\raggedright
\bibliography{paper}
\endgroup

\end{document}

%                       Revision History
%                       -------- -------
%  Date         Person  Ver.    Change
%  ----         ------  ----    ------

%  2013.06.29   TU      0.1--4  comments on permission/copyright notices

