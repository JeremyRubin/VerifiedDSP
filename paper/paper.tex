%-----------------------------------------------------------------------------
%
%               Template for sigplanconf LaTeX Class
%
% Name:         sigplanconf-template.tex
%
% Purpose:      A template for sigplanconf.cls, which is a LaTeX 2e class
%               file for SIGPLAN conference proceedings.
%
% Guide:        Refer to "Author's Guide to the ACM SIGPLAN Class,"
%               sigplanconf-guide.pdf
%
% Author:       Paul C. Anagnostopoulos
%               Windfall Software
%               978 371-2316
%               paul@windfall.com
%
% Created:      15 February 2005
%
%-----------------------------------------------------------------------------


\documentclass[preprint,12pt]{sigplanconf}


\usepackage{amsmath}
\usepackage{listings}
\usepackage{amsfonts}
\usepackage{color}
\usepackage{url}
\usepackage{hyperref}

\definecolor{mygreen}{rgb}{0,0.6,0}
\definecolor{mygray}{rgb}{0.5,0.5,0.5}
\definecolor{mymauve}{rgb}{0.58,0,0.82}
\definecolor{mygray2}{rgb}{1,1,1}

\lstset{ %
  backgroundcolor=\color{mygray2},   % choose the background color; you must add \usepackage{color} or \usepackage{xcolor}
  basicstyle=\footnotesize,        % the size of the fonts that are used for the code
  breakatwhitespace=false,         % sets if automatic breaks should only happen at whitespace
  breaklines=true,                 % sets automatic line breaking
  captionpos=b,                    % sets the caption-position to bottom
  commentstyle=\color{mygreen},    % comment style
  deletekeywords={...},            % if you want to delete keywords from the given language
  escapeinside={\%*}{*)},          % if you want to add LaTeX within your code
  extendedchars=true,              % lets you use non-ASCII characters; for 8-bits encodings only, does not work with UTF-8
  %% frame=single,                    % adds a frame around the code
  keepspaces=true,                 % keeps spaces in text, useful for keeping indentation of code (possibly needs columns=flexible)
  keywordstyle=\color{blue},       % keyword style
  language=Octave,                 % the language of the code
  morekeywords={*,...},            % if you want to add more keywords to the set
  numbers=left,                    % where to put the line-numbers; possible values are (none, left, right)
  numbersep=5pt,                   % how far the line-numbers are from the code
  numberstyle=\tiny\color{mygray}, % the style that is used for the line-numbers
  rulecolor=\color{black},         % if not set, the frame-color may be changed on line-breaks within not-black text (e.g. comments (green here))
  showspaces=false,                % show spaces everywhere adding particular underscores; it overrides 'showstringspaces'
  showstringspaces=false,          % underline spaces within strings only
  showtabs=false,                  % show tabs within strings adding particular underscores
  stepnumber=1,                    % the step between two line-numbers. If it's 1, each line will be numbered
  stringstyle=\color{mymauve},     % string literal style
  tabsize=2,                       % sets default tabsize to 2 spaces
  title=\lstname                   % show the filename of files included with \lstinputlisting; also try caption instead of title
}
%%
%% Coq definition (c) 2001 Guillaume Dufay
%% <Guillaume.Dufay@sophia.inria.fr>
%% with some modifications by J. Charles (2005)
%% And Jeremy Rubin (2015)
%%
\lstdefinelanguage{Coq}%
                  {morekeywords={Variable,Inductive,CoInductive,Fixpoint,CoFixpoint,%
                      Definition,Example,Lemma,Theorem,Axiom,Local,Save,Grammar,Syntax,Intro,%
                      Trivial,Qed,Intros,Symmetry,Simpl,Rewrite,Apply,Elim,Assumption,%
                      Left,Cut,Case,Auto,Unfold,Exact,Right,Hypothesis,Pattern,Destruct,%
                      Constructor,Defined,Fix,Record,Print,Proof,Induction,Hint,Hints,Exists,
                      let,in,Parameter,Split,Red,Reflexivity,Transitivity,if,then,else,Opaque,%
                      Transparent,Inversion,Absurd,Generalize,Mutual,Cases,of,end,Analyze,%
                      AutoRewrite,Functional,Scheme,params,Refine,using,Discriminate,Try,%
                      Require,Load,Import,Scope,Set,Open,Section,End,match,with,Ltac,%
                      exists,forall,fix,fun,Prop,Type,Module
                    },%
                    sensitive, %
                    morecomment=[n]{(*}{*)},%
                    morestring=[d]'',%
                    literate={=>}{{$\Rightarrow$}}1 {>->}{{$\rightarrowtail$}}2{->}{{$\to$}}1
                    {\/\\}{{$\wedge$}}1
                    {|-}{{$\vdash$}}1
                    {\\\/}{{$\vee$}}1
                    {~}{{$\sim$}}1
                    %{<>}{{$\neq$}}1 indeed... no.
                  }[keywords,comments,strings]%


\begin{document}

\special{papersize=8.5in,11in}
\setlength{\pdfpageheight}{\paperheight}
\setlength{\pdfpagewidth}{\paperwidth}

\conferenceinfo{6.888 '15}{\today, Cambridge, MA, U.S.A.} 
\copyrightyear{2015} 
\copyrightdata{978-1-nnnn-nnnn-n/15/03} 
\doi{nnnnnnn.nnnnnnn}


\titlebanner{6.888: Certified Systems Software 2015}        % These are ignored unless

\preprintfooter{Framework to prove 8-bit embedded
  software systems on the Intel 8051 series} % 'preprint' option specified.
\title{VerifiedDSP: Verifying Digital Signal Processing Designs in Coq}
\subtitle{\url{https://github.com/JeremyRubin/VerifiedDSP}}

\authorinfo{Jeremy Rubin}
 {jlrubin@mit.edu}

\maketitle

\begin{abstract}
  Digital Signal Processors and microcontrollers are used widely in a
wide range of devices and machines.  There are many life critical
applications, including medical equipment, transport, and
communications. It is therefore of great importance to ensure the
proper functionality of such devices. Many of these devices are
simple, preferring to rely on a tried and true 8-bit architecture
without a full operating system so that they may more easily reason
about real-time responses to events. For instance, it could be
disastrous to have a garbage collection pause while trying to apply
the brakes of a car. However, this simplicity comes at a cost; without
higher level constructs a programmer must manually write and check a
lot more code due to the high resource constraints. Furthermore, it is
hard for a programmer to verify that the code they wrote is correctly
translated into the binary loaded onto chip, the compiled version may
have different properties than desired.


   This paper presents a new framework, VerifiedDSP, for programming
8-bit Intel 8051 series microcontrollers and designing embedded
Digital Signal Processing systems. It includes an 8051 simulator, a
prototyping framework for mocking out specifications, and some higher
level constructs to help programmers formalize the behavior and run
time of control loops.
\end{abstract}

%% \category{CR-number}{subcategory}{third-level}

% general terms are not compulsory anymore, 
% you may leave them out
%% \terms
%% term1, term2

%% \keywords
%% keyword1, keyword2

\section{Introduction}
Modern machines often rely on a combination of mechanical systems,
electrical systems, sensors, and software. The advent of embedded
systems has allowed for much more accurate and complicated designs to
be built because they allow for tighter feedback loops as well as
convenient ways to interface between disparate systems.

Often times these embedded systems are built with simple
microcontrollers ($\mu Cs$) such as 8-bit because a more powerful
computers are not needed, energy-inefficient, and are harder to reason
about for real-time applications. In some sectors, Real-Time Operating
Systems (RTOS) have become popular due to programming ease, but there
are many applications where a lighter weight solution is needed. Many
systems which require light weight microcontrollers also happen to be
life-critical. Unfortunately, there have been many instances where the
life-critical nature of this role was made all too clear with
catastrophic failures.

This paper presents an effort to reduce the likelihood of engineering
errors as well as toolchain errors when building a Digital Signal
Processing unit (DSP), including complex designs using a
microcontroller such as the 8-bit Intel 8051 series
microcontrollers\footnote{This chip was chosen because the author is
familiar with them and they are very widely deployed.}.

There are two main components of this project: a
trusted-but-proof-ready 8051 simulator, and utilities which assist
programmers in prototyping formally reasoning about behavior of
embedded designs. Behavior is characterized by traces on ``pins''.

These traces can encode many properties such as return values,
execution time, pin output, and non interference.

The internal behavior of the microcontroller is provable, but the
higher level specifications of a design, only deal in traces. The
internal state of the microcontroller can still be reasoned about for
proofs, but it is masked to allow for a better development
cycle. Microcontroller functionality can be mocked out, and refinement
can be shown between a functional description and a microcontroller.

This system will have many limitations which may affect its real-world usefulness,
which will be expounded in the discussion.

\section{Motivation}
There are several concrete examples of embedded system failures which
motivate this work.
\subsection{Medical Devices}
The Therac-25 was a radiation therapy machine which lethally overdosed
6 people \cite{therac}. While a large portion of the of the blame for
this could be placed on the lack of hardware failsafe, it was
ultimately unverified software which caused the deadly malfunction.

Another interesting medical device is the pacemaker. A 2004 study saw
that approximately 1/1,200 pacemaker reprogramming events loaded faulty
programs onto the device, which could cause a patient's heart rate to
elevate to 185 beats per minutes. The fault was due to multiple
timer interrupts being enabled at the same time, when only one was
supposed to be enabled at any given time \cite{pacemaker}.
\subsection{Transport}
Automobiles are also a large concern with regards to embedded systems
as they rely on multiple for the core functionality of the
vehicle. Over the past decade, there have been a number of
bugs and errors caused -- or suspected to be caused -- by faulty
software.  For instance, several major automobile manufacturers --
Nissan, Honda, and Subaru -- recalled vehicles for faulty embedded
systems which could deploy airbags during normal vehicle operation or
start the car at random while parked \cite{cars}.

\subsection{Communications}
Networked mobile devices use small hardware secure chips called
subscriber identification module cards (SIM cards) to reliably
identify clients connecting to the network. They contain a $\mu C$
which manages the processing of cellular data, such as encryption and
signatures for requests from a certain number (such as send a
text-message). Recently, exploits have been found that allow a remote
attacker to take control of the SIM card and issue illegitimate
requests \cite{sim}.
\section{Design Overview}
The following subsections will contain an overview of the major parts of
this project: \autoref{subsec:component} will discuss the component
definitions; \autoref{subsec:wires} will discuss the wiring framework
used to connect the components; \autoref{subsec:running} discusses the
execution of a wiring;\autoref{subsec:8051} will discuss the microcontroller implementations.


\subsection{Component} \label{subsec:component}

The fundamental definition in VerifiedDSP is the \emph{IO} module, which
has 3 main definitions detailed in \autoref{lst:iomod}.
\begin{lstlisting}[language=Coq,label={lst:iomod},caption={IO module definition}]
Module IO.
  Definition t :=  nat.
  Definition trace := list (list t).
  Inductive func :=
  | fn_args : nat -> (trace -> t) -> func.
End IO.
\end{lstlisting}

\paragraph{IO.func}
The most important definition is IO.func. An IO.func is a pair of a
number of arguments and a function which can process an IO.trace to produce
an output IO.t.
\paragraph{IO.t} IO.t can be thought of as a \emph{voltage}, and it is the type passed
around by other modules.
\paragraph{IO.trace}
The IO.trace is simply a list of lists of the IO.t voltage. Each trace is
implicitly numbered by their position in the top list.
\paragraph*{}
\begin{lstlisting}[language=Coq, caption={An
integration unit, an incrementing unit, and an always 0 ``rail''},
label={lst:simplecomp}]
  Definition integrator : IO.func :=
      IO.fn_args 1 (fun x =>suml (hd [] x)).
  Definition incrementor : IO.func := 
      IO.fn_args 1 (fun x =>len (hd [] x)).
  Definition zero_rail : IO.func := 
      IO.fn_args 0 (fun _ =>0).
\end{lstlisting}
Some example components are shown in \autoref{lst:simplecomp}.

\subsection{Wiring}\label{subsec:wires}
At the core of this project is a way of connecting these components
together to perform computations. This
is modeled as an infinite breadboard with a single discrete time clock. There is a set of
$\mathbb{N}$ pins that a device can connect to. Devices can read any
pin, and can write to a single pin. Multiple outputs can be simulated
by making duplicate copies of the same hardware which write to
different pins. \autoref{lst:wiring} shows the breadboard definition.
\begin{lstlisting}[language=Coq, label={lst:wiring}, caption={wiring
type definition}]
  Inductive wiring :=
  | base :  wiring
  | watch_set :  wiring -> list nat ->
                 IO.func -> nat ->
                 wiring
  | just_set : wiring  -> 
               IO.func -> nat -> wiring
  | join : wiring -> wiring -> wiring
  | doc : wiring -> nat -> string -> wiring.
\end{lstlisting}
watch\_set observes a list of pins, and writes to a single pin.\\
just\_set only has access to a single clock trace, but sets values without observing other state.\\
join joins two separate networks together.\\
doc adds non functional documentation for convenience, which is useful for more complex projects.\\
We define the following notations for the wiring:
\begin{equation}
w //  m \sim> f \sim>  n \equiv watch\_set\ w\ m\ f\ n 
\end{equation}
\begin{equation}
w */  m \sim> f \sim>  n \equiv just\_set\ w\ f\ n 
\end{equation}
\begin{equation}
  w1\ \sim\&\sim\ w2 \equiv join\ w1\ w2
\end{equation}
\begin{equation}
  w\ \#\ p\ c  \equiv doc\ w\ p\ c
\end{equation}


Once a wiring is complete, it should be checked for correctness. The
property valid\_wiring, as described in \autoref{lst:validwire},
checks this. These invariants are important for a correct design.


\begin{lstlisting}[language=Coq, label={lst:validwire}, caption={Check that all pins that are
being read from are written to, that there is no contention for pins
(ie, two devices driving the same pin, and that all devices are fully
connected to the proper number of pins.)}]
Fixpoint valid_wiring' w ins outs: Prop :=
  match w with 
    | base  =>
      set_intersect ins outs = ins
    | w' // from ~> fn_args n f ~> to =>
      ~set_in to outs /\
      n = length from /\
      valid_wiring' w' (union ins from)
          (set_add to outs)
    | w' */ fn ~> to=>
      ~set_in to outs
      /\ valid_wiring' w' ins
             (set_add to outs)
    | w1 ~&~ w2 =>
      let w1o := output_pins w1 [] in
      let w2o := output_pins w2 [] in
      valid_wiring' w1 (union w2o ins)
          (union w2o outs) /\
      valid_wiring' w2  (union w1o ins)
          (union w1o outs)
    | w' # _ _ => 
        valid_wiring' w'  ins outs
  end.
Definition valid_wiring w := valid_wiring' w nil nil.
\end{lstlisting}

Wirings are not correct by construction. This is because it is
desirable to produce incomplete modules which need to be \emph{plugged in} to one
another (ie, using a join) before being finished. The doc constructor is 
important for this so that developers may easily check to see what the
other developer intended to be connected.

If two modules are needed in the same system, but they conflict there is a 
function $rewire$ which provably reconnects a system so as not to
conflict on any pin. It works by adding the max pin number + 1 of the
other circuit to every pin numbering. If the smallest pin number in a
system is greater than the largest in the other, then it is clear that
they will not clash. The specific implementation should not be relied
as many different rewirings could be performed. Understanding
rewirings is another use for the documentation functionality.


\begin{lstlisting}[language=Coq, caption={A simple wiring}, label={lst:simplewire}]
Definition example :=
    base 
    */  zero_rail ~> 0
    */  incrementor ~> 2
    //  [2] ~> integrator ~> 3
    #   3 "Integrated incrementor".
\end{lstlisting}

Two modules which obeys valid\_wiring can not interfere with one another and
can be safely joined. A valid wiring can be joined with an arbitrary
(valid or invalid)
wiring an either become invalid or stay valid, if it stays valid then
certain properties should be preserved as shown in \autoref{lst:nonint}.


\begin{lstlisting}[language=Coq, caption={On adding a new module w\'\ to a valid wired w, for all traces the outputs are
the same if present, or may be added if not present.} , label={lst:nonint}]
Theorem non_interference1 :
forall w w', 
valid_wiring w ->
valid_wiring  (w ~&~ w') ->
forall n t,
let orig := find_trace t (run w n) in
let mod := find_trace t (run (w ~&~ w') n) in
match  orig, mod with
    | Some a, Some b => a = b
    | Some a, None => False
    | None, Some a => True
    | None, None => True
end.
\end{lstlisting}

\subsection{Running}\label{subsec:running}
Once a wiring is constructed, it can be \emph{run}. Running a wiring
involves walking over the wire structure and applying the latest
traces to determine the next states. Functions are computed over the
entire trace at each step. This could be optimized, but for proofs it
is simpler. The run function can be proven to preserve history (\autoref{lst:preserve}) and
present consistent views of pins to all functions in the structure.
For example, the trace on pin 3 from \autoref{lst:simplewire} is 
 [36; 28; 21; 15; 10; 6; 3; 1; 0; 0].
 

\begin{lstlisting}[language=Coq, label={lst:preserve}, caption={The
next run is identical up to the latest result}]
Theorem no_modify_history: forall n w,  run  w n = tl (run w (S n)).
\end{lstlisting}


  
\subsection{8051}\label{subsec:8051}

An IO.func can be implemented as any function which can operate over a IO.trace.
Thus, it is possible to design significantly more complicated IO.funcs
than those in \autoref{lst:simplecomp}. VerifiedDSP includes a general purpose 8051 IO.func that can be integrated
into any design. The basic signature of the device is as follows:

\begin{lstlisting}[language=Coq, caption={The adc and dac provide
conversions to and from the networks type, the bin is a list of
bytes to be loaded into code memory. The 8051's ports must all be connected. First the program is loaded into
memory, then the component can be executed over any set of 32 pin
traces. There is only one output pin, but as noted earlier duplicate
devices can be used to simulate multiple output pins.}]
Definition i8051_Component 
  bin adc dac :=
  fn_args (8*4) (fun t =>
                   let ps := traces t adc in
                   dac (run_8051_bin_string bin ps)).
\end{lstlisting}


A specific program on an 8051 can be shown to simulate another
function using the relation func\_same in \autoref{lst:fsame}. 

\begin{lstlisting}[language=Coq, label={lst:fsame}, caption={Check
that over any well formed IO.trace, two IO.funcs have identical
results. In Theorem simulates, an 8051 is shown to simulate a zero function} ]
Definition func_same (i i':IO.func) := 
    forall (tr:IO.trace),
    let lengths := (map
    (fun x => length x)
    tr) in 
    let fl := hd 0 lengths
    in
    fold_left (fun acc  x
    =>x=fl/\acc)  lengths
    True->
    match i, i' with
      |IO.fn_args n f,
      IO.fn_args n' f'=>
       length tr = n ->
       f tr = f' tr /\ n = n'
    end.

(** 
[2;0;0] is 8051 binary for:
.org 0h
main:
ljmp main
 **)
Theorem simulates :  func_same (i8051_Component [2;0;0] threshold dac)
(IO.fn_args (8*4) (fun _ => 0)).
\end{lstlisting}


\subsubsection{Implementation and Progress}
A complete VerifiedDSP 8051 model is still in progress. Currently, the
8051 can run approximately 5 instructions (eg, JMP, LJMP, SETB, CLR,
ANL). The implementations are heavliy derived from RockSalt
\cite{rocksalt}, and is basically a port from their x86 model to
8051. \autoref{lst:i8051} shows the RTL implementation of the SETB instruction.


\begin{lstlisting}[language=Coq, label={lst:i8051}, caption={The SETB
instructions Register Transfer Logic implementation. The argument type is first
checked, and then the argument is checked to be a valid bit address
(not all values are bit addressable). The proper bit is or'ed into memory.}]
Definition conv_SETB (op1:operand) : Conv unit :=
match op1 with
    | Bit_op (bit_addr baddr) =>
      if is_valid_bit_addr baddr then
        let bsel := and baddr 3 in
        let addr := and baddr (~3) in
        let ormask := shl 1 bsel in
        ormaskReg <- load_int ormask;
        a <- load_int addr;
        v <- read_byte a;
        v' <- arith or_op v ormaskReg;
        write_byte v' a
        else
          emit error_rtl
    | _ => emit error_rtl
end.
\end{lstlisting}

The semantics of the processor are basically correct, remaining work
would be in implementing the remainder of the instruction set and
developing proof automation to make it easier to develop with.


\subsection{Extraction}
As a last note, the designs can be extracted into running programs in
Ocaml using the Coq extraction facilities. This is of dubious value
given that mostly the correctness of designs is interesting (and can
thus be done from within Coq), but it could be useful for running
tests on prototype designs before trying to formally prove them.

\section{Related Work}
Several projects have related goals:

\subsubsection{Bedrock}
Bedrock is a project which facilitates the formal verification of low
level code. It is implemented as a Coq Library, but it provides
significant enough extensions and capabilities that it is more aptly
described as a language for separation logic. Bedrock provides
significant instruction on the implementation of a separation logic
system\cite{bedrock1}\cite{bedrock2}.
\subsubsection{Reflex}
Reflex is a project which makes a verified message passing kernel that
can interpret a set of user defined rules about non interference and
causality. Reflex is not intended for embedded systems, but the model
of non interference and causality is instructive for a style in which
one might prove properties about an embedded system \cite{reflex}. For
example, one such property could be that seems natural to try to prove
in a Reflex style could be that at least 100 cycles pass in between
asking for and reading a value from an Analog Digital Converter.
\subsubsection{Correctness Proofs for Device Drivers in Embedded Systems}
Duan and Regher\cite{drivers} discuss a system which can verify
low-level device drivers.  Their approach allows for separately
clocked components interactions to be modeled and verified.  They use
a preexisting ARMv4 simulator and develop a formal semantics for
interacting with UART. Their proofs are all manual, and they hope to
build proof automation on top of their proofs for UART.
\subsubsection{RockSalt}
This effort derives heavily from RockSalt. RockSalt was a project to
write a formally proven version of Native Client(NaCl), Google's tool
for generating and checking untrusted binaries which can then be
safely run with confidence they will never escape the sandbox
\cite{rocksalt}.

A major part of the RockSalt contribution was an excellent model of
Register Transfer Language on top of which they modeled MIPS and x86
processors. The ability to end-to-end check properties of the binary
is a powerful tool in guaranteeing functional correctness. Their
effort was very instructive in the implementation of the 8051
model; it is an adaptation of their code.
\subsubsection{Model Checking Software for Microcontrollers}
This project formally models an 8-bit Atmega microcontroller.
The project reasons about assembly code, but is designed to have proof
automation for high level theorems from C, C++, and other programming langauges.
There does not seem to be facilities for run time verifition \cite{mcsm}.

\subsubsection{An Approach for the Formal Verification of DSP Designs
  using Theorem Proving}
This project is very similar in nature to VerifiedDSP. They
implemented a similar architecture using HOL. However, their
implementation does not have a full microcontroller, just a Register
Transfer Logic. They prove correct an FFT algorithm in this manner,
and then are able to export it into a netlist for hardware synthesis.
They seem to lack an elegant method of connecting modules together,
unlike the wiring construct in VerifiedDSP.  There is no source code
available for their implementation\cite{dspprove}.

\section{Discussion}
\subsection{Future Work}


\subsubsection{Usability Study}
This framework cannot prevent all classes of bugs. In order to test
the effectiveness of such a system (once there is more automation), it
is imperative to run a user study to see if such a model helps. One
such user study could testing and timing users on their ability to
complete the following tasks with and without the simulation tools
(given a small hardware test-bench):
\begin{enumerate}
\item
  Write a program which makes an LED blink with a certain frequency
\item
  Write a program which writes a copy of an array into a specific
  memory location, maps $f(x) = x+1$ over the array, without modifying
  any other memory.
\item
  Write a program which when a push button is hit, turns one LED on at
  half brightness, and another LED at half brightness otherwise.
\end{enumerate}
These tests would determine if these tools improves a programmer's
ability to reason about runtime, non-interference, and memory safety.



\subsubsection{Limitations}
The current design fails to address several key
embedded system properties and has a few weaknesses.

\paragraph{Simulator Correctness}
The simulator is not guaranteed to be correct. In fact, it is
definitely not correct as certain aspects of the model are not fully
implemented, such as setting ports IO mode. The model can be proven to
be consistent within itself (ie, no confused parse), but there is no
way to check that the model matches the real world 8051. A test-jig
could be built to verify the 8051 simulator against real hardware.

\paragraph{Interrupts}
Interrupts are explicitly required to be disabled because they would
add a lot of additional complexity to the system. One could potentially,
using an 8051 timers, devise a scheme as follows:
\vfill
\begin{lstlisting}[caption={The timer ISR is used to guaranteed a bound on the amount of time interrupt might be executing.}]
  timer isr:
  disable interrupts
  go to interrupts_over
  set timer to priority high with size 200 cycles
  set all other interrupts priority low
  enable interrupts
  some interruptible loop
  interrupts_over:
\end{lstlisting}
Such a scheme allows for a provably bounded window on interrupts; this
is of dubious value given that polling would also accomplish a similar
goal with more deterministic performance.

\paragraph{More Hardware Interfaces}
Embedded systems fundamentally interface with hardware, which can have
difficult to understand behavior. This work only has the semantics 
for an 8051 and some simple DSP designs. However, one exciting possibility would be the
development of semantics for a library peripheral chips.  This would allow
programmers to verify more complicated and realistic designs.

Furthermore, an additional modulus
could be added  onto modules to support slower clock cycle than the rest of
the network, which may be desirable to simulate communicating with
faster/slower hardware.

\paragraph{Nondeterminism}
This work does not help with verification given unreliable hardware
such as sensors or power supply. 

\paragraph{Assembler}

Another challenge would be to implement an assembler so that code in
higher level expressions can be reliably compiled. Making an assembler in
Coq may have some other benefits in aiding later verifiable macro
development \cite{coqasm}. Currently, the standard 8051 as31 compiler
is used.

\subsection{Reflections on Formal Verification}
This was my first experience with formal verification. In sum, formal
verification is hard, very hard.


\subsubsection{Code Reuse}
One of the great promises of formal verification is code reuse. Once
algorithms or software are proven correct once, they can become a permanent
fixture and a basis for more formalized software. In theory, at
least. In practice, I found it very difficult to work with existing
code bases for a number of reasons. First and foremost, it seems that
the Coq compiler has breaking changes fairly frequently, as I was
unable to build fairly recent code from the RockSalt project. The
reason for this was, it seems, that Coq makes major tradeoffs in the
proof scripts in terms of variable names. Another factor which made it
difficult for me was just the overall complexity of the existing
code. Hacking on something as complex as RockSalt really frustrated
my progress, when I started developing the more novel parts of this
project I was able to make a lot more progress because I knew what I 
wanted and was trying to do. Furthermore, I discovered a bug in the
RockSalt code and would be interested to see if it exists in the
main code as well and was not a result of my modifications.
\begin{lstlisting}[language=Coq, caption={The RockSalt bug: the parser
helper for immediate constants only reads the MSB of the constant}]
  Definition bad_immediate_16 := 
  (field 8) @ ((fun r => Imm16_op
  (@Word.repr 15 r)) :
      _ -> result_m operand_t).

  Definition immediate_16 := 
  (field 16) @ ((fun r => Imm16_op
  (@Word.repr 15 r)) :
      _ -> result_m operand_t).
\end{lstlisting}

\subsubsection{Abandon Proof!}
Stating Theorems is much easier than proving them, and from my
experience, a large portion of the benefit of formal verification
is simply stating the theorems as it forces a deep reflection on the
architecture and functionality of the code. That said, once I had
Theorems, proving them seemed to be very difficult, and I abandoned
many proofs. However, in the process of trying to prove them (and
failing) I discovered some more critical bugs which prevented any
progress from proceeding at all, which suggests that trying
to prove things is useful even if you give up! I also found that
development went more quickly when I posed the theorems then admitted
them as I built more architecture.


\subsubsection{What Did I Just Prove}
Getting the higher level specifications correct in a verification effort is
difficult. What is trying to be built? What semantics should it have?
When the all the proofs are done, does it actually accomplish the
desired goal? This is difficult to know.

\subsubsection{Editors}
I was a vim user before I started writing Coq code. Luckily evil-mode
lessened the transition burden to emacs, but certainly a new editor
was a large distraction!

\subsubsection{Time}
I found it hard to hack lightly on this stuff. Whereas with most
projects I can make fine progress with an hours time, I almost never
got anything done within the first three contiguous hours of work on
this project. This high startup cost made making progress during a
busy semester hard. However, scheduling large blocks of time to work
on it was ultimately effective.

\subsubsection{Interactivity}
This is a really great UI feature for programming, and I'm surprised
that more languages don't have an interactive compilation mode like
Coq's. Perhaps it wouldn't be useful in other languages, but I found
being able to quickly go line by line to be a very intuitive way to
code compared to my traditional workflow. 

\subsubsection{Overview}
Despite the obstacles faced, I really enjoyed doing this
project. Writing Coq code and proofs is insanely addicting; I would
routinely find myself at 5 in the morning telling myself to go to
sleep. It is challenging, and sometimes I would spend hours trying to
prove something trivial. Almost nothing compared to the exhilaration
I felt when I got a tough proof Qed'd though! The puzzle solving
nature really does something for me. 

Writing Coq is certainly not like any other traditional programming
language, including the functional varieties such as Haskell or
Ocaml. Although it has its negatives, such as poor reusability of code
and high barriers to entry, the positives outweighed it by far for
me. Simply put, it's just a lot of fun!


\subsection{Conclusion}
This paper presents a novel framework, VerifiedDSP, which can be used
to verify and prototype Digital Signal Processing designs in Coq. This
framework is very flexible, as it can serve as a backbone for
prototyping digital signal processors because functionality can be
mocked out and then refined with closer to hardware function
descriptions (ie, microcontroller binaries or RTL). 

Additionally, this paper details the author's first major experience
with formal verification techniques. There were several obstacles and
tactics to get around them that might be good advice for someone just
starting to explore the field, and can hopefully guide future work in
making formal methods more accessible.






%% \appendix
%% \section{Appendix Title}

%% This is the text of the appendix, if you need one.

\acks
A huge thank you to professors Frans Kaashoek and Nickolai Zeldovich
for teaching the Certified Systems seminar at MIT. It has been one of
the most interesting classes I have taken at MIT by far! Without their
incessant encouragement and support, I would not have been able to complete
this project.

\bibliographystyle{abbrvnat}
\begingroup
\raggedright
\bibliography{paper}
\endgroup

\end{document}

%                       Revision History
%                       -------- -------
%  Date         Person  Ver.    Change
%  ----         ------  ----    ------

%  2013.06.29   TU      0.1--4  comments on permission/copyright notices

